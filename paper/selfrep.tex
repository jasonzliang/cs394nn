%f='selfrep'; b='selfrep'; pdflatex $f.tex && bibtex $f && pdflatex $f.tex && pdflatex $f.tex && rm $f.log $f.aux && evince $f.pdf &>/dev/null &disown



\documentclass[12pt]{article}
\usepackage[margin=1in]{geometry}
\usepackage[protrusion=true,
            expansion=true]{microtype}
%\usepackage{amssymb}
%\usepackage{amsmath}
\usepackage{booktabs}
\usepackage{color}
\usepackage[usenames,
            dvipsnames]{xcolor}
\usepackage{graphicx}
\usepackage{caption}
\usepackage{subcaption}

\usepackage{kpfonts}
\usepackage[T1]{fontenc}
\usepackage{setspace}
\singlespacing
%\onehalfspacing
%\doublespacing



\newcommand{\term}[1]{\emph{#1}}



\begin{document}

\title{Programs that Program}
\author{Keenan Breik. Jason Liang}
\date{}
\maketitle

\section*{Introduction}

Evolutionary computation allows computers
to automatically solve problems
that can be cast as optimization problems.
Until now, instantiations have been hand designed.

We propose to allow computers
to automatically design such instantiations
by using evolutionary computation itself.
To do so, we desire programs
that write other programs
and thereby explore a search space.
In this paper, we demonstrate
that neural networks
(which can behave as programs or program components)
can generate other meaningful neural networks.

\cite{stanley2002neat}

\bibliography{selfrep}
\bibliographystyle{plain}

\end{document}
